\documentclass[journal,12pt,twocolumn]{IEEEtran}
%
\usepackage{setspace}
\usepackage{gensymb}

\singlespacing

\usepackage[cmex10]{amsmath}

\usepackage{amsthm}
\usepackage[latin1]{inputenc}
\usepackage{mathrsfs}
\usepackage{txfonts}
\usepackage{stfloats}
\usepackage{bm}
\usepackage{cite}
\usepackage{cases}
\usepackage{subfig}
\usepackage{karnaugh-map}
\usepackage{longtable}
\usepackage{multirow}

\usepackage{enumitem}
\usepackage{mathtools}
\usepackage{steinmetz}
\usepackage{tikz}
\usepackage{circuitikz}
\usepackage{verbatim}
\usepackage{tfrupee}
\usepackage[breaklinks=true]{hyperref}

\usepackage{tkz-euclide}

\usetikzlibrary{arrows, shapes.gates.logic.US, calc}
\usepackage{listings}
    \usepackage{color}                                            %%
    \usepackage{array}                                            %%
    \usepackage{longtable}                                        %%
    \usepackage{calc}                                             %%
    \usepackage{multirow}                                         %%
    \usepackage{hhline}                                           %%
    \usepackage{ifthen}                                           %%
    \usepackage{lscape}     
\usepackage{multicol}
\usepackage{chngcntr}
%\usepackage{tfrupee}
                                       %%
  %optionally (for landscape tables embedded in another document): %%
    \usepackage{lscape}     
\usepackage{multicol}
\usepackage{chngcntr}
%\usepackage{enumerate}
\usepackage{karnaugh-map}

%\usepackage{wasysym}
%\newcounter{MYtempeqncnt}
\DeclareMathOperator*{\Res}{Res}
%\renewcommand{\baselinestretch}{2}
\renewcommand\thesection{\arabic{section}}
\renewcommand\thesubsection{\thesection.\arabic{subsection}}
\renewcommand\thesubsubsection{\thesubsection.\arabic{subsubsection}}

\renewcommand\thesectiondis{\arabic{section}}
\renewcommand\thesubsectiondis{\thesectiondis.\arabic{subsection}}
\renewcommand\thesubsubsectiondis{\thesubsectiondis.\arabic{subsubsection}}

% correct bad hyphenation here
\hyphenation{op-tical net-works semi-conduc-tor}
\def\inputGnumericTable{}                                 %%

\lstset{
%language=C,
frame=single, 
breaklines=true,
columns=fullflexible
}
%\lstset{
%language=tex,
%frame=single, 
%breaklines=true
%}

\begin{document}
%


\newtheorem{theorem}{Theorem}[section]
\newtheorem{problem}{Problem}
\newtheorem{proposition}{Proposition}[section]
\newtheorem{lemma}{Lemma}[section]
\newtheorem{corollary}[theorem]{Corollary}
\newtheorem{example}{Example}[section]
\newtheorem{definition}[problem]{Definition}
%\newtheorem{thm}{Theorem}[section] 
%\newtheorem{defn}[thm]{Definition}
%\newtheorem{algorithm}{Algorithm}[section]
%\newtheorem{cor}{Corollary}
\newcommand{\BEQA}{\begin{eqnarray}}
\newcommand{\EEQA}{\end{eqnarray}}
\newcommand{\define}{\stackrel{\triangle}{=}}
\bibliographystyle{IEEEtran}
%\bibliographystyle{ieeetr}
\providecommand{\mbf}{\mathbf}
\providecommand{\pr}[1]{\ensuremath{\Pr\left(#1\right)}}
\providecommand{\qfunc}[1]{\ensuremath{Q\left(#1\right)}}
\providecommand{\sbrak}[1]{\ensuremath{{}\left[#1\right]}}
\providecommand{\lsbrak}[1]{\ensuremath{{}\left[#1\right.}}
\providecommand{\rsbrak}[1]{\ensuremath{{}\left.#1\right]}}
\providecommand{\brak}[1]{\ensuremath{\left(#1\right)}}
\providecommand{\lbrak}[1]{\ensuremath{\left(#1\right.}}
\providecommand{\rbrak}[1]{\ensuremath{\left.#1\right)}}
\providecommand{\cbrak}[1]{\ensuremath{\left\{#1\right\}}}
\providecommand{\lcbrak}[1]{\ensuremath{\left\{#1\right.}}
\providecommand{\rcbrak}[1]{\ensuremath{\left.#1\right\}}}
\theoremstyle{remark}
\newtheorem{rem}{Remark}
\newcommand{\sgn}{\mathop{\mathrm{sgn}}}
\providecommand{\abs}[1]{$\left\vert#1\right\vert$}
\providecommand{\res}[1]{\Res\displaylimits_{#1}} 
\providecommand{\norm}[1]{$\left\lVert#1\right\rVert$}
%\providecommand{\norm}[1]{\lVert#1\rVert}
\providecommand{\mtx}[1]{\mathbf{#1}}
\providecommand{\mean}[1]{E$\left[ #1 \right]$}
\providecommand{\fourier}{\overset{\mathcal{F}}{ \rightleftharpoons}}
%\providecommand{\hilbert}{\overset{\mathcal{H}}{ \rightleftharpoons}}
\providecommand{\system}{\overset{\mathcal{H}}{ \longleftrightarrow}}
	%\newcommand{\solution}[2]{\textbf{Solution:}{#1}}
\newcommand{\solution}{\noindent \textbf{Solution: }}
\newcommand{\cosec}{\,\text{cosec}\,}
\providecommand{\dec}[2]{\ensuremath{\overset{#1}{\underset{#2}{\gtrless}}}}
\newcommand{\myvec}[1]{\ensuremath{\begin{pmatrix}#1\end{pmatrix}}}
\newcommand{\mydet}[1]{\ensuremath{\begin{vmatrix}#1\end{vmatrix}}}
\makeatletter
\@addtoreset{figure}{problem}
\makeatother
\let\StandardTheFigure\thefigure
\let\vec\mathbf
%\renewcommand{\thefigure}{\theproblem.\arabic{figure}}
\renewcommand{\thefigure}{\theproblem}
%\setlist[enumerate,1]{before=\renewcommand\theequation{\theenumi.\arabic{equation}}
%\counterwithin{equation}{enumi}
%\renewcommand{\theequation}{\arabic{subsection}.\arabic{equation}}
\def\putbox#1#2#3{\makebox[0in][l]{\makebox[#1][l]{}\raisebox{\baselineskip}[0in][0in]{\raisebox{#2}[0in][0in]{#3}}}}
     \def\rightbox#1{\makebox[0in][r]{#1}}
     \def\centbox#1{\makebox[0in]{#1}}
     \def\topbox#1{\raisebox{-\baselineskip}[0in][0in]{#1}}
     \def\midbox#1{\raisebox{-0.5\baselineskip}[0in][0in]{#1}}
\vspace{3cm}
\title{
%	\logo{
EE5811 : FPGA LAB \\ ASSIGNMENT 1
%	}
}
\author{ Anuradha Uggi (EE21RESCH01008)}	
% make the title area
\maketitle
\newpage
%\tableofcontents
\bigskip
\renewcommand{\thefigure}{\theenumi}
\renewcommand{\thetable}{\theenumi}
%\renewcommand{\theequation}{\theenumi}
Download the codes from
\begin{lstlisting}
https://github.com/Anuradha-Uggi/FPGA-LAB-2022/blob/main/A1/A1.c
\end{lstlisting}
\section{\textbf{PROBLEM STATEMENT}}
Reduce the following Boolean Expression to its simplest form using K-Map.
\begin{align}
    F(P,Q,R,S)=\sum(0,1,2,3,5,6,7,10,14,15)
\end{align}
\section{\textbf{SOLUTION}}
Using K-Map \ref{fig:kmap}, simplified SOP expression is:
\numberwithin{figure}{section}
\begin{figure}[h]
\centering
\begin{karnaugh-map}[4][4][1][$$RS$$][$$PQ$$]
    \minterms{0,1,2,3,5,6,7,10,14,15}
    \maxterms{4,8,9,11,12,13}
    \implicant{0}{2}
    \implicant{2}{10}
    \implicant{1}{7}
    \implicant{7}{14}
    
    \draw[color=black, ultra thin] (0, 4) --
    node [pos=0.7, above right, anchor=south west] {$RS$} % YOU CAN CHANGE NAME OF VAR HERE, THE $X$ IS USED FOR ITALICS
    node [pos=0.7, below left, anchor=north east] {$PQ$} % SAME FOR THIS
    ++(135:1);
    
\end{karnaugh-map}
\caption{Karnaugh-Map}
\label{fig:kmap}
\end{figure}
\begin{align}
    F(P,Q,R,S) &= \sum(0,1,2,3,5,6,7,10,14,15) \\
    &= \Bar{P}\Bar{Q} + R\Bar{S} + \Bar{P}S+QR
\end{align}
\subsection{Using Nand Logic:}
\begin{align}
F &=\Bar{P}\Bar{Q} + R\Bar{S} + \Bar{P}S+QR\\
  &=((\bar{P}\bar{Q})'(P\bar{S})'(\bar{P}S)'(QR)')'
\end{align}

Now we can draw the logic circuit using NAND gates as below.\\\\
\begin{figure}[hbt!]
    \centering
\begin{circuitikz}[label distance=2mm, scale=2,
  connection/.style={draw,circle,fill=black,inner sep=1.5pt}
  ]
\node (x) at (0.5,0) {$P$};
\node (y) at (1,0) {$Q$};
\node (z) at (1.5,0) {$R$};
\node (w) at (2.0,0) {$S$};
\node[nand gate US, draw, rotate=0, logic gate inputs=inni, scale=1.5] at ($(z)+(1.5,-1)$) (t1) {};
\node[nand gate US, draw, rotate=0, logic gate inputs=nnni, scale=1.5] at ($(z)+(1.5,-2)$) (t2) {};
\node[nand gate US, draw, rotate=0, logic gate inputs=innn, scale=1.5] at ($(z)+(1.5,-3)$) (t3) {};
\node[nand gate US, draw, rotate=0, logic gate inputs=nnnn, scale=1.5] at ($(z)+(1.5,-4)$) (t4) {};
\node[nand gate US, draw, rotate=0, logic gate inputs=nnnn, scale=1.5] at ($(t2)+(1.5,-0.5)$) (t5) {};

\draw (x) -- ($(x) + (0,-5.5)$);
\draw (y) -- ($(y) + (0,-5.5)$);
\draw (z) -- ($(z) + (0,-5.5)$);
\draw (w) -- ($(w) + (0,-5.5)$);

\draw (x) |- (t1.input 1) node[connection,pos=0.5]{};
\draw (y) |- (t1.input 4) node[connection,pos=0.5]{};

\draw (x) |- (t2.input 1) node[connection,pos=0.5]{};
\draw (w) |- (t2.input 4) node[connection,pos=0.5]{};

\draw (x) |- (t3.input 1) node[connection,pos=0.5]{};
\draw (w) |- (t3.input 4) node[connection,pos=0.5]{};

\draw (y) |- (t4.input 1) node[connection,pos=0.5]{};
\draw (z) |- (t4.input 4) node[connection,pos=0.5]{};

\draw (t1.output) -- ([xshift=0.3cm]t1.output) |- (t5.input 1);
\draw (t2.output) -- ([xshift=0.2cm]t2.output) |- (t5.input 2);
\draw (t3.output) -- ([xshift=0.2cm]t3.output) |- (t5.input 3);
\draw (t4.output) -- ([xshift=0.3cm]t4.output) |- (t5.input 4);
\draw (t5.output) -- ([xshift=0.3cm]t5.output) node[above]{$F$};

\end{circuitikz}
\caption{Logic Circuit using NAND gates}
\label{ckt1}
\end{figure}
\end{document}
